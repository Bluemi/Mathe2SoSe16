\documentclass[fleqn,a4paper]{scrartcl}

\usepackage{fullpage}
\usepackage[utf8]{inputenc}
\usepackage{lmodern}
\usepackage{amsmath}
\usepackage{amssymb}
\usepackage{amsfonts}
\usepackage{pgfplots}
\usepackage{tikz}
\usepackage{pgf}

\begin{document}
\[
\int \frac{dx}{(x-a)^n} = \int (x-a)^{-n}dx = \frac{(x-a)^{1-n}}{1-n} +c = \frac{1}{(1-n)(x-a)^{n-1}} +c \] \\ für \[ n>1 \] \\ Bsp \[ 
f(x) = \frac{x^2 + 2x +1}{(x-1)(x-2)^2} = \frac{(x+1)^2}{(x-1)(x-2)^2} \] \\ \[
Q(x) = (x-1) (x-2)^2 \] \\ \[
f(x) = \frac{A}{x-1}+ \frac{B}{x-2}+\frac{C}{(x-2)^2} = \frac{1}{(x-1)(x-2)^2}[A(x-2)^2 + B(x-1)(x-2) +C(x-1)]\] \\ \[
=\frac{1}{N} [A(x^2-4x+4)+B(x^2-3x+2)+C(x-1)] \] \\ \[
=\frac{1}{N}[x^2(A+B) +x(-4A-3B+c)+4A+2B-C] \] \\ 
\begin{eqnarray*}
1: 4A+2B-C =& 1 \qquad \vert + \\
x: -4A-3B+C =& 2\qquad \vert + \\
x^2: A+B =& 1 \\
-B =& 3 \\
\end{eqnarray*}
\begin{eqnarray*}
B=-3& \\
A=4& \qquad \qquad \qquad f(x) = \frac{4}{x-1} -\frac{3}{x-2} + \frac{9}{(x-2)^2} \\
C=9&
\end{eqnarray*}
\[
\int f(x) dx = 4 \int \frac{dx}{x-1} -3 \int \frac{dx}{x-2} + 9 \int \frac{dx}{(x-2)^2} \] \\ \[
= 4 \ln \vert x-1 \vert -3 \ln \vert x-2 \vert - \frac{9}{x-2} +c \] \\ Fall: Nenner enthaelt einen Faktor $ x^2 + 2px + q $ , der keine reellen Nullstellen besitzt, d.h. $ p^2 -q < 0 $ \\ \[ x_{1,2} = -p \quad \underset {-} {+} \sqrt{p^2 -q} \] \\ Ansatz für die Partialbruchzerlegung \\ \[ \frac{Bx+C}{x^2+2px+q} \] \\ \[
\int \frac{dx}{x^2+1} = \arctan x +c \] \\ \[
\int \frac{dx}{x^2+a^2} = \frac{1}{a} \arctan \frac{x}{a} +c \] \\ denn \[ \int \frac{dx}{x^2 + a^2} = \int \frac{dx}{a^2 ((\frac{x}{a})^2 + 1 )} = \frac{1}{a} \frac{dt}{t^2+1} \bigg \vert _{t=\frac{x}{a} } \] \\ \[ t= \frac{x}{a} \rightarrow dt = \frac{dx}{a} \] \\ \[ = \frac{1}{a} \arctan t \bigg \vert _{t=\frac{x}{a}} +c = \frac{1}{a} \arctan \frac{x}{a} +c \] \\ \[
p^2 -q < 0 : q-p^2>0 \] \\ \[
x^2+2px+q = (x+p)^2 +q -p^2 =x^2+2px+p^2+ \underbrace {q-p^2}_{q} \] \\ \[ 
I = \int \frac{dx}{x^2+2px+q} = \int \frac{dx}{(x+p)^2 + \underbrace{q-p^2}_{a^2}} \] \\ \[
t= \frac{x+p}{a} \rightarrow dt = \frac{dx}{a} = \frac{dx}{\sqrt{q-p^2}} \] \\ \[ 
I= \frac{1}{\sqrt{q-p^2}} \int \frac{dt}{t^2+1} \bigg \vert _{t=\frac{x+p}{a}} = \frac{1}{\sqrt{q-p^2}} \arctan \frac{x+p}{\sqrt{a-p^2}} +c $ \\ falls $ q-p^2 >0 \]  Bsp \[ 
\int \frac{dx}{x^2-4x+5} = \int \frac{dx}{(x-2)^2+1} \] \\ \[
p=-2, q=5 \rightarrow a= \sqrt{q-p^2} = \sqrt{5-(-2)^2} = 1 \] \\ \[
\int \frac{dx}{x^2-4x+5} = \arctan(x-2) +c
\]
\end{document}