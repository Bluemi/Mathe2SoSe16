\documentclass{article}

\usepackage[utf8x]{inputenc}
\usepackage[T1]{fontenc}
\usepackage{fullpage}
\usepackage{amsmath}
\usepackage{amssymb}
\usepackage{amsfonts}
\usepackage{pxfonts}
\usepackage{ucs}
\usepackage{hyperref}
\usepackage{xfrac}
\usepackage{lmodern}
\usepackage{graphicx}
\usepackage{multirow}
\usepackage{tikz}
\usepackage{pgfplots}
\usepackage{pgf}

% Schreibweisen für bestimmte Überschriften:
%
%					Beispiele
% \section{große Überschrift} 		Folgen und Reihen
% \subsection{große Überschrift} 	Häufungspunkte und Grenzwerte von Folgen
% \paragraph{Definition}
% \paragraph{Schreibweise}
% \paragraph{Bemerkung}
% \paragraph{Bezeichnung}
% \paragraph{Satz}

\begin{document}

% zwischen den beiden folgenden kommentaren schreiben

% start
$\Rightarrow f(a-x)+f(a+x)=2b=2f(a)$
\begin{itemize}
\item Periodizität\\
$f$ periodisch mit Periode $\lambda$, falls $f(x)=f(x+\lambda)$\\
Folge: $f(x)=f(x+k\lambda), k \in \mathbb{Z}$
\end{itemize}
\begin{enumerate}
  \setcounter{enumi}{3}
  \item Verhalten am Rand des Definitionsbereichs und an den Polstellen.\\
  \begin{tikzpicture}[domain=-0.2:4]
	\draw[<->] (-0.2,0) -- (4,0) node[right] {$x$};
	\draw[<->] (0,-1) -- (0,2) node[above] {$y$};
	\draw[domain=0:2][color=red] plot[samples=100] function{1.5*(x-1)**2+0.2};
	\draw[domain=2:3.5][color=blue] plot[samples=100] function{-0.5*(x -3)**10};
	\draw[dashed] (2,1.7) -- (2,-0.5);
\end{tikzpicture}\\
  \item Unstetigkeitsstellen 
  \item Achsenabschnittpunkte\\
  $f(0)$, insbesondere Nullstellen
  \item Monotonie und Extremwerte
  \item Krümmungsverhalten und Wendepunkte
  \item Asymptoten\\
  Einie Funktion $g(x)$ heißt Asymptote zu $f(x)$ für 
  $x \rightarrow x_0$ $x_0 \in \mathbb{R} \cup \{-\infty, \infty\}$
  falls entweder
  \begin{itemize}
  \item $g(x)<f(x)$ oder $g(x)>f(x)$\\
  in einem Intervall oder für ein Halbegrade
  \item $|f(x)-g(x)| \rightarrow 0$ für $x \rightarrow x_0$
  \end{itemize}
\end{enumerate}

\section{Kurvendiskussion}
\begin{enumerate}
 \item Definitionsbereich\\
 Zuordnungsvorschrift $f: x \mapsto f(x)$\\
 Wie sieht der maximale Definitionsbereich aus?
 \item Symmetrie und Periodizität
 \begin{itemize}
  \item Achsensymmetrie Spiegelung an $f(a-x) = f(a+x)$ $x = a$\\
  speziell: gerade Funktion $f(x)=f(-x)$
  \item Punktsymmetrie am Punkt $(a, f(a))$\\
  $f(a-x) + f(a+x) = 2f(a)$\\
  spezielle: ungerade Funktion $f(x)=-f(-x)$\\
  Bsp: $f(x) = b + \sin(x - a)$\\
  \begin{tikzpicture}[domain=10:-0.5]
    \draw[<->] (-0.5,0) -- (7.5,0) node[right] {$x$};
    \draw[<->] (0,-0.5) -- (0,2.5) node[above] {$y$};
    \draw[domain=0.5:7][color=red] plot[samples=100] function{sin(x - 3.5)+1.5};
    \draw[solid] (3.5,1.8) -- (3.5,1.2);
    \draw[solid] (3.2,1.5) -- (3.8,1.5);
    \draw[solid] (3.5,0.1) -- (3.5,-0.1) node[below] {$b$};
    \draw[solid] (-0.1,1.5) -- (0.1,1.5) node[left] {$a$};
\end{tikzpicture}\\
  $f(a-x)=b+\sin((a-x)-a)=b+\sin(-x)=b-\sin x$\\
  $f(a+x)=b+\sin((a+x)-a)=b+\sin x$
 \end{itemize}

\end{enumerate}

% stop

\end{document}
