\documentclass{article}

\usepackage{fullpage}
\usepackage{amsmath}
\usepackage{amssymb}
\usepackage{tikz}

% Schreibweisen für bestimmte Überschriften:
%
%					Beispiele
% \section{große Überschrift} 		Folgen und Reihen
% \subsection{große Überschrift} 	Häufungspunkte und Grenzwerte von Folgen
% \paragraph{Definition}
% \paragraph{Schreibweise}
% \paragraph{Bemerkung}
% \paragraph{Bezeichnung}
% \paragraph{Satz}

\begin{document}

% zwischen den beiden folgenden kommentaren schreiben

% start
\paragraph{gesucht $y = \tanh^{-1} x$} 
\begin{align*}
x=& \tanh y = \frac{e^{2y}-1}{e^{2y}+1} &\bigg \vert (e^{2y} +1)\\
(e^{2y}+1)x=&e^{2y}-1\\
e^{2y}(x-1)=&-1-x\\
e^{2y}(1-x)=&1+x\\
e^{2y}=&\frac{1+x}{1-x}\\
\tanh^{-1} x =& y = \frac{1}{2} \ln \frac{1+x}{1-x}
\end{align*}
\paragraph{analog $\coth^{-1} x = \frac{1}{2} \ln \frac{x+1}{×-1}$}
\paragraph{Ableitungen:}
\begin{align*}
\sinh^{-1'}  x =& \frac{1}{\sqrt{x^{2} + 1}}\\
\cosh^{-1'}  x =& \frac{1}{\sqrt{x^{2} - 1}}, |x| > 1\\
\tanh^{-1'}  x =& \frac{1}{1 - x^{2}}, |x| < 1\\
\coth^{-1'}  x =& +\frac{1}{1- x^{2}}, |x| > 1\\
\end{align*}

% ar tanh x -1 bis 1
\begin{tikzpicture}[domain=2.5:-2.5]
	\draw[<->] (-2.5,0) -- (2.5,0) node[right] {$x$};
	\draw[<->] (0,-2.5) -- (0,2.5) node[above] {$y$};
	\draw[domain=-2:2][color=red] plot[samples=100] function{0.5 * log((1+x)/(1-x))}
		node[right] {$\tanh^{-1} x$};
	\draw[dashed] (-1,-2.5) -- (-1,0);
	\draw[dashed] (1,0) -- (1,2.5);
	\draw (1,0.1) -- (1,-0.1) node[below] {$1$};
	\draw (-1,0.1) -- (-1,-0.1) node[above] {$-1$};
\end{tikzpicture}

% coth x -1 bis 1
\begin{tikzpicture}[domain=2.5:-2.5]
      	\draw[<->] (-2.5,0) -- (2.5,0) node[right] {$x$};
	\draw[<->] (0,-2.5) -- (0,2.5) node[above] {$y$};
	\draw[domain=-2:2][color=red] plot[samples=100] function{tanh(x)}
		node[right] {$\tanh x$};
	\draw[domain=-2:-0.5][color=red] plot[samples=100] function{1/tanh(x)}
		node[right] {$\coth x$};
	\draw[domain=0.5:2][color=red] plot[samples=100] function{1/tanh(x)};
	\draw[dashed] (0,1) -- (2.5,1);
	\draw[dashed] (-2.5,-1) -- (0,-1);
	\draw (0.1,1) -- (-0.1,1) node[left] {$1$};
	\draw (0.1,-1) -- (-0.1,-1) node[right] {$-1$};
\end{tikzpicture}

% ar coth -2 bis -1 und 1 bis 2
\begin{tikzpicture}[domain=2.5:-2.5]
	\draw[<->] (-2.5,0) -- (2.5,0) node[right] {$x$};
	\draw[<->] (0,-2.5) -- (0,2.5) node[above] {$y$};
	\draw[domain=-2:-1][color=red] plot[samples=100] function{0.5 * log((x+1)/(x-1))};
	\draw[domain=1:2][color=red] plot[samples=100] function{0.5 * log((x+1)/(x-1))}
		node[right] {$\coth^{-1} x$};
	\draw[dashed] (-1,-2.5) -- (-1,0);
	\draw[dashed] (1,0) -- (1,2.5);
	\draw (1,0.1) -- (1,-0.1) node[below] {$1$};
	\draw (-1,0.1) -- (-1,-0.1) node[above] {$-1$};
\end{tikzpicture}

\paragraph{gesucht $y= \sinh^{-1} x$}
\begin{align*}
x =& \sinh y = \frac{1}{2} (e^y\pm e^{-y})\\
e^y \pm e^{-y} - 2 x =& 0\\
e^{2y} \pm 1 - 2 x e^y =& 0\\
z =& e^y > 0\\
z^{2} - 2xz \pm 1 =& 0\\
z_{1/2} =& x \pm \sqrt{x^{2} \pm 1} = e^y\\
\Rightarrow& y = sinh^{-1} x = \ln (x + \sqrt{x^{2}} \pm 1)
\end{align*}
% stop

\end{document}
