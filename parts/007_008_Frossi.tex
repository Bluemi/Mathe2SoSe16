\documentclass{article}

\usepackage{fullpage}
\usepackage{amsmath}
\usepackage{amssymb}
\usepackage{tikz}
\usepackage[utf8]{inputenc}

% Schreibweisen für bestimmte Überschriften:
%
%					Beispiele
% \section{große Überschrift} 		Folgen und Reihen
% \subsection{große Überschrift} 	Häufungspunkte und Grenzwerte von Folgen
% \paragraph{Definition}
% \paragraph{Schreibweise}
% \paragraph{Bemerkung}
% \paragraph{Bezeichnung}
% \paragraph{Satz}

\begin{document}

% zwischen den beiden folgenden kommentaren schreiben

% start
alternativ
\begin{align*}
f(x) &= \frac {(x+3)^2}{(x-1)^3}\\\\
f'(x) &= \frac{[(x+3)^2]'(x-1)^3-(x+3)^2[(x-1)^3]'}{(x-1)^6}\\\\
&= \frac{2(x+3)*1(x-1)^3 - (x+3)^2 *[3(x-1)^2 *1]}{(x-1)^6}\\\\
&= \frac{(x+3)(x-1)^2[2(x-1)-(x+3)*3]}{(x-1)^6}\\\\
&= \frac{x+3}{(x-1)^4} * 2x-2-3x-9\\\\
&= \frac{(x+3)(-x-11)}{(x-1)^4}
\end{align*}

\subsection{Newton-Verfahren}

Gesucht \(x^2\) mit \(f(x^2)=0\)\\\\
Nullstellen der Tangente
\begin{align*}
0\overset{!}{=}y(x_t+1) &= f(x_t)+(x_t+1-x_t)f'(x_t)\\\\
\Longrightarrow x_t+1 &= x_t - \frac{f(x_t)}{f'(x_t)}
\end{align*}
Satz:\\
In einem Intervall \(I \le \mathbb{R}\) sei die \(f:\mathbb{R} \rightarrow \mathbb{R}\) zweimal stetig differenzierbar und es gebe ein \(a \in I\) mit \(f(a)=0\). Es gelte \(M := \underset{x \in I}{sup} |f'(x)| < \infty\) und \(m := \underset{x \in I}{inf} |f'(x)| > 0\)\\
Für einen Startwert \(x_0 \in I\) und ein \(d = |x_0 -a|\) gelte \([a-d,a+d] \leq I\) und \(q = \frac{Md}{2m} < 1\). Dann wird durch \(x_{t+1} = x_t - \frac{f(x_t)}{f'(x_t)}\) rekursiv eine Folge \((x_t) |x_t -a|<d\) definiert.\\\\
a-priori-Abschätzung \(|x_n -a| \leq \frac{2m}{M} * q^{2^n}\)  \((n \in \mathbb{N})\)\\\\
a-postteriori-Abschätzung \(|x_n -a| \leq \frac{M}{2m}|x_n - x_{x+1}|^2\)\\\\
Bsp: \(f(x) = \sqrt{x}-4+x\)\\\\
Nullstellen \(a \in [0,4]\)\\\\
Nährung \(\widetilde{x} : |f(\widetilde{x})| < 10^{-2}\)

% stop

\end{document}