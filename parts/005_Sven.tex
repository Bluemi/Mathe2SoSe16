\documentclass{article}

\usepackage{fullpage}
\usepackage{amsmath}
\usepackage{amssymb}

\usepackage{ucs}
\usepackage[utf8x]{inputenc}
\usepackage[T1]{fontenc}

% Schreibweisen für bestimmte Überschriften:
%
%					Beispiele
% \section{große Überschrift} 		Folgen und Reihen
% \subsection{große Überschrift} 	Häufungspunkte und Grenzwerte von Folgen
% \paragraph{Definition}
% \paragraph{Schreibweise}
% \paragraph{Bemerkung}
% \paragraph{Bezeichnung}
% \paragraph{Satz}

\begin{document}

% zwischen den beiden folgenden kommentaren schreiben

% start

gilt auch für \(\alpha = 0\)

\begin{align*}
	f(x) = x^\alpha\\
	= x^0\\
	= 1
\end{align*}
für \(x > 0\)

\begin{align*}
	\alpha x^{\alpha-1} &= 0*x^{-1}\\
	&= 0
\end{align*}
für \(\alpha = 0\)

\begin{equation*}
	f'(x) = [1]' = 0
\end{equation*}

\subsection*{Bsp2)}

\begin{equation*}
	f(x) = x^x\\
\end{equation*}
für \(x > 0\)

\begin{align*}
	F(x) &= \ln|f(x)|\\
	&= \ln|x^x|\\
	&= x\ln x
\end{align*}

\begin{align*}
	F'(x) &= x'\ln x+x\ln' x\\
	&= 1\ln x+x\frac{1}{x}\\
	&= \ln x+1
\end{align*}

\begin{align*}
	f'(x) &= f(x)F'(x)\\
	&= x^x(1+\ln x)
\end{align*}

$
	f(x) = 1
$

% stop

\end{document}
