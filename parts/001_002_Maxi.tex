\documentclass{article}

\usepackage{fullpage}
\usepackage{amsmath}
\usepackage{amssymb}
\usepackage{pxfonts}
\usepackage{tikz}
% Schreibweisen für bestimmte Überschriften:
%
%					Beispiele
% \section{große Überschrift} 		Folgen und Reihen
% \subsection{große Überschrift} 	Häufungspunkte und Grenzwerte von Folgen
% \paragraph{Definition}
% \paragraph{Schreibweise}
% \paragraph{Bemerkung}
% \paragraph{Bezeichnung}
% \paragraph{Satz}
 
% zwischen den beiden folgenden kommentaren schreiben


\begin{document}
% start
\section{Logarithmus - Funktion}
 

Seien $a,b > 0$ Dann bezeichnet $x = log\textsubscript{a}$ die Loesung der Gleichung $a\textsuperscript{x} = b$
\\
\\Spezielle Bezeichnungen:
\\$a = 2 : y = lb x = log\textsubscript{2}\ x \leftrightarrow x = 2\textsuperscript{y} 
\\a = 10 : y = lg x = log\textsubscript{10}\ x \leftrightarrow x = 10\textsuperscript{y}
\\a = e : y = ln x = log\textsubscript{e}\ x \leftrightarrow x = e\textsuperscript{y}
$

\subsection{Regeln:}


$log\textsubscript{a}\ a = 1\\
\\log\textsubscript{a}\ 1 = 0\\
\\log\textsubscript{a}\ (x \cdot y) = log\textsubscript{a}\ x + log\textsubscript{a} y\\
\\log\textsubscript{a}\ (\frac{x}{y}) = log\textsubscript{a}\ x - log\textsubscript{a} y\\
\\log\textsubscript{a}\ x\textsuperscript{Beta} = Beta\ log\textsubscript{a}\ x (Beta \in R)\\
\\speziell\ log\textsubscript{a}\ \sqrt[n]{x} = log\textsubscript{a}\ x\textsuperscript{ $\frac{1}{n}$ } = \frac{1}{n} log\textsubscript{a}\ x\\
$

\subsection{Umrechnung der Logarithmen zu verschiedenen Basen a, b $>$ 0}

$
\\z\textsubscript{1}\ = log\textsubscript{a}\ x \leftrightarrow a\textsuperscript{z\textsubscript{1}}\ = x\\
\\z\textsubscript{2}\ = log\textsubscript{b}\ x \leftrightarrow b\textsuperscript{z\textsubscript{2}}\ = x\\
\\a\textsuperscript{z\textsubscript{1}}\ = x = b\textsuperscript{z\textsubscript{2}}\\
\\log\textsubscript{a}\ a\textsuperscript{z\textsubscript{1}}\ = z\textsubscript{1}\ \underbrace{log\textsubscript{a}\ a}_{1}\ = z\textsubscript{1}\ = log\textsubscript{a}\ b\textsuperscript{z\textsubscript{2}}\ = z\textsubscript{2}\ log\textsubscript{a}\ b \\
\\
$[$log\textsubscript{b}\ x = z\textsubscript{2}\ = 
\frac{z\textsubscript{1}}{log\textsubscript{a}\ b}\ = 
\frac{log\textsubscript{a}\ x}{log\textsubscript{a}\ b}
$]$
$


\subsection{Differentration}

$y = ln\ |x|, x \neq\ 0\ $daraus folgt$\ y' = \frac{1}{x}$
\\
\begin{tikzpicture}
	\draw[->] (-2.5,0) -- (2.5,0) node[right] {$x$};
	\draw[->] (0,-2.5) -- (0,2.5) node[above] {$y$};
	\draw (-0.8,-1) arc (0:80:3cm);
	\draw (0.8,-1) arc (180:100:3cm);
\end{tikzpicture}
$
\\
\\f(-x)= f(x)\\
$

$\\
\\
$

\begin{tikzpicture}
	\draw[->] (-2.5,0) -- (2.5,0) node[right] {$x$};
	\draw[->] (0,-2.5) -- (0,2.5) node[above] {$y$};
	\draw (2.5,1) arc (270:190:2cm);
	\draw (-1,-2.5) arc (0:80:2cm);
\end{tikzpicture}
$
\\
\\f'(-x)= -f'(x)\\
\\
$
(1)$f(-x)=f(x) \rightarrow\ f'(-x)=-f'(x)\\
$
(2)$f(-x)=-f(x) \rightarrow\ f'(-x)= f'(x)\\
$
\\
$
(1)f(x)=f(g(x))=(f \circ g)(x)$ daraus folgt $(f \circ g )'(x)= f'(y) \cdot g'(x)$ bei $y = f(x)\\
$

$\\g(x)= -x, g'(x)= -1 = -f'(-x)$\\
 
% stop

\end{document}